\documentclass{ximera}
\title{Installing Locally}

\begin{document}
\begin{abstract}
Instructions for installing Ximera locally
\end{abstract}
\maketitle

First you will want to install ``Ubuntu'' for Windows. You download
this from the Microsoft Store. While this process can be somewhat
complex, however, following the prompts will get you there.

Once  you get this, launch Ubuntu you will see a window like the following:


Go ahead and press any key. If the window abruptly closes, restart
Windows and try again


\link[Ximera]{http://ximera.osu.edu} can also be run on your own machine rather than through CoCalc. Note that you will need to initialize your own git repos if you are not running through the CoCalc xandbox using \verb!git init!.

\section{Install ximeraLaTeX}

This can be done by going to \link[XimeraProject/ximeraLatex GitHub
  page]{https://github.com/XimeraProject/ximeraLatex}, and scrolling
down to the directions for your particular platform.




\section{Install xake}

Xake is Ximera's version of ``make'' and it converts the
\LaTeX\ source into HTML.

\end{document}
