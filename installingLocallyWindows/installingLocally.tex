\documentclass{ximera}
\title{Installing Locally: Windows}

\begin{document}
\begin{abstract}
Instructions for installing Ximera locally.
\end{abstract}
\maketitle

First you will want to install ``Ubuntu'' for Windows. You download
this from the Microsoft Store. While this process can be somewhat
complex, however, following the prompts will get you there.

At some point it will ask you for a username and password. Choose a
username that is reasonable, and a password that you can type when
needed (it doesn't need to be super secret).

Once it is installed you will see something like:

\begin{verbatim}
your-user-name@your-computers-name:~$
\end{verbatim}

Now lets keep your computer up-to-date:

\begin{verbatim}
sudo apt update
sudo apt upgrade
\end{verbatim}

Now we need to install some software:

\begin{verbatim}
sudo apt install texlive-full 
\end{verbatim}



\section{Install ximeraLaTeX}

This can be done by going to \link[XimeraProject/ximeraLatex GitHub
  page]{https://github.com/XimeraProject/ximeraLatex}, and scrolling
down to the directions for your particular platform.


\section{Install xake}

Xake is Ximera's version of ``make'' and it converts the
\LaTeX\ source into HTML.



\end{document}
