\documentclass{ximera}
\title{Installing locally: Windows}

\begin{document}
\begin{abstract}
Instructions for installing Ximera locally.
\end{abstract}
\maketitle

For Windows, we will install the Ximera document class, and Ubuntu for Windows.

Then the work-flow is that one creates documents on the Windows side,
uses GitHub to push them to a remote repository, and then pulls to
Ubuntu for Windows. Finally \verb|xake| is used to deploy the
documnets to the web on the Ubuntu Side.


\section{Install ximeraLaTeX}


Visit the repo on Github.com \link[XimeraProject/ximeraLatex GitHub
  page]{https://github.com/XimeraProject/ximeraLatex} and click the
``Clone in Desktop'' button.



Create the directory structure:

\begin{verbatim}
C:\localtexmf\tex\latex\
\end{verbatim}

and move ximeraLatex to \verb|C:\localtexmf\tex\latex\|. For MiKteX to notice this
directory, go to:

\begin{itemize}
\item Start $\to$ All programs $\to$ MiKTeX Folder $\to$ Maintenance (Admin) Folder $\to$ Settings (Admin).
\item Now select the tab ``Roots.''
\item Click ``Add'' because you are going to add a path.
\item Find \verb|C:\localtexmf\| and click ``OK.''
\item Click ``apply'' then ``OK.''
\item Reopen Miktex Settings (Admin). Click Refresh FNDB.
\end{itemize}
This will allow all of your documents to find \verb|ximera.cls|. Note, it is
important that none of the directories containing ximeraLatex have
spaces in their names. Once you move ximerLatex, you will need to tell
your GitHub client where to look for it, you can do this using ``find
files.''





\section{Installing Ubuntu for windows}





First you will want to install ``Ubuntu'' for Windows. You download
this from the Microsoft Store. While this process can be somewhat
complex, however, following the prompts will get you there.

At some point it will ask you for a username and password. Choose a
username that is reasonable, and a password that you can type when
needed (it doesn't need to be super secret).

Once it is installed you will see something like:

\begin{verbatim}
your-user-name@your-computers-name:~$
\end{verbatim}

Now lets keep your computer up-to-date:

\begin{verbatim}
sudo apt update
sudo apt upgrade
\end{verbatim}

Now we need to install some software:

\begin{verbatim}
sudo apt install texlive-full 
\end{verbatim}



\section{Install ximeraLaTeX}

This can be done by going to \link[XimeraProject/ximeraLatex GitHub
  page]{https://github.com/XimeraProject/ximeraLatex}, and scrolling
down to the directions for your particular platform.


\section{Install xake}

Xake is Ximera's version of ``make'' and it converts the
\LaTeX\ source into HTML.



\end{document}
