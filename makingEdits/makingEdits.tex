\documentclass{ximera}
%\def\sectionautorefname~#1\null{\S#1\null}
%\def\subsectionautorefname~#1\null{\S#1\null}
\title{Making Edits}
%\def\itemautorefname~#1\null{#1\null}
\begin{document}
\begin{abstract}
Instructions for editing an existing activity file.
\end{abstract}
\maketitle


\subsection{Editing existing activity files}
This is a tiny change to see if it shows up. Maybe it will, but I doubt it.
\begin{enumerate}

\item Click on ``Files" to view the files that have been downloaded into your project.
\item Choose the directory ``xandbox" from the list of files.
\item Choose ``first.tex"
\item Make some edits and click ``Save."
\item Click on the terminal in your project.
\item Type \verb!cd xandbox! to enter your xandbox directory.
\item Type \verb!xake name! followed by a space and then a short lowercase name for your xandbox.  This could be your first name, for instance.  The name you choose must be globally unique, so be creative!  For instance, \verb!xake name turnloon! is what I will choose.
\item Type \verb!xake bake! to compile your first.tex into an html file.  If you run into errors, you can go back to your first.tex file and make additional edits.
\item insert step here
\item Type \verb!git add first.tex! to stage the changes you've made to the first.tex file.
\item Type \verb!git commit -m "My first edit"! to commit the staged changes.
\item Type \verb!xake frost! to create a publication commit on top of your source commit.
\item Type \verb!xake serve! to share your content with the world.  For instance, my content will appear at https://ximera.osu.edu/yourxakename/first
\end{enumerate}
\subsection{Stuck?}

If you are stuck, please contact us at ximera@math.osu.edu to get help.


\end{document}
