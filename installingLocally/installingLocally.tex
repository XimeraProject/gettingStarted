\documentclass{ximera}
\title{Installing Locally}

\begin{document}
\begin{abstract}
Instructions for installing Ximera locally
\end{abstract}
\maketitle


\link[Ximera]{http://ximera.osu.edu} can also be run on your own machine rather than through CoCalc. Note that you will need to initialize your own git repos if you are not running through the CoCalc xandbox using \verb!git init!.

\subsection{Terminal commands for installing xake on Ubuntu 16.04}
\begin{verbatim}
sudo apt-get install golang-go
\end{verbatim}
Corresponding libgit2 and git2go installations are needed:
\begin{verbatim}
sudo apt-get install libgit2-dev
sudo apt-get install golang-git2go-dev
mkdir -p ~/go/src/github.com/ximeraproject
export GOPATH=$HOME/go
cd ~/go/src/github.com/ximeraproject
git clone https://github.com/XimeraProject/xake.git
cd ~/go/src/github.com/libgit2/git2go
git checkout v24
cd ~/go/src/github.com/ximeraproject/xake
go get .
go build .
Use command ./xake or better yet, add xake to PATH by...
Also set your PATH so that ~/go/bin/ is in it: (edit .profile)
PATH=$PATH:~/go/bin
\end{verbatim}
Reboot!  Now you should be able to use the command xake

\subsection{Terminal commands for installing xake on Mac}
\begin{enumerate}

\item First you need \link[HomeBrew]{https://brew.sh}. Follow the link and the instructions there.
\item Then you need to install \link[Go]{http://www.golangbootcamp.com/book/get_setup}. Note: the first instruction is to install HomeBrew, which you have already done. 
\item You might need to install ''pkg-config” and ''libgit2”. This can be done via HomeBrew:
\begin{verbatim}
brew install pkg-config
brew install libgit2
brew install gpg
brew install mupdf
\end{verbatim}
\item Now perform the following commands in terminal:
\begin{verbatim}
mkdir -p ~/go/src/github.com/ximeraproject
export GOPATH=$HOME/go
cd ~/go/src/github.com/ximeraproject
git clone https://github.com/XimeraProject/xake.git
cd xake
go get .
go build .
\end{verbatim}

\item If you don’t already have a GPG key, then do
\verb!gpg --gen-key!, answer the questions, and copy the long hex string as YOUR-GPG-KEY-ID ( ABCD3562DBF9929292 or whatever)
\begin{verbatim}
gpg --keyserver hkps://ximera.osu.edu/ --send-key YOUR-GPG-KEY-ID

PATH=$PATH:~/go/bin

xake -k YOUR-GPG-KEY-ID name yourxakename
\end{verbatim}
You will need to enter your passkey; prompt may be broken but try typing it and hitting enter anyway

\item You may get vague error messages here if you have not successfully set up your git name and email; you can do this with:
\begin{verbatim}
git config --global user.name "NAME”
git config --global user.email “EMAIL”
\end{verbatim}
\end{enumerate}
\end{document}
