\documentclass{ximera}

\title{How to Download Data}

\begin{document}
\begin{abstract}
  Using xake data.
\end{abstract}
\maketitle

\section{Downloading Data}

You can download the data from your students' work using the \verb!xake data! command.

To download the data, run \verb!xake data download!. This will save your data as a log.sz file.

\subsection{Viewing and Exporting Data}

To see your data, you can run either \verb!xake data csv! to access a csv version of the data, or \verb!xake data json! to access a json version. The json version contains much more material.

To save the output as a file, use e.g. \verb!xake data json > myData.json!.

\section{Analyzing Data}

Analyzing these data can be difficult because there is so much. If you use R for analysis, the \verb!jsonlite! package will import the xake data export into a data frame.

Before loading the json file, open it in a text editor. You will see "This is xake, Version 0.9.2" at the beginning with a line break. Delete these so a square bracket is the first line. On the last line, there will be a last comma that needs to be deleted before the final square bracket. Delete this and save the file.

If your data has been saved as \verb!myData.json!, you can use \verb!myData <- fromJSON("myData.json", flatten=TRUE)! to save the data as a dataframe.

\subsection{Variables}

The most relevant variables for analysis are:
\begin{itemize}
\item actor: identifies the person until the session has ended. A person may have multiple actor codes depending on whether they have logged it or not, or used different computer, etc.
\item stored: the time when the entry occurred. Timestamp is based on the user's computer, and maybe or off by hours depending on how their computer is set up. stored is a better choice for the time
\item verb.display.en-US: the verb for what was happening at that record
\begin{itemize}
\item experienced: a page loaded
\item answered: a question was answered
\item completed: something was completed (not sure what qualifies here)
\item played: a video was played
\item watched: a video was watched from some time to another time
\item skipped: a video was skipped from one time to another time (after being paused)
\end{itemize}
\item object.definition.name.en-US: the name of the page loaded
\item result.response: whatever was entered into an answer box
\item result.success: if the question is graded, this is \verb!TRUE! if that answer was correct and \verb!FALSE! if the answer was incorrect.
\end{itemize}

\end{document}
