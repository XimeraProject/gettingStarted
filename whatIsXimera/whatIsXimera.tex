\documentclass{ximera}

\graphicspath{
{./}
{../whatIsXimera/}
{../settingUpTheRepository/}
{../sageMathCloud/}
}


\title{What is Ximera?}
\outcome{Understand what Ximera is and how it can be used}
\begin{document}
\begin{abstract}
An introduction to the Ximera system.
\end{abstract}
\maketitle


\section{Introduction}

\link[Ximera]{http://ximera.osu.edu} (pronounced \textipa{[k\textscripta ImI\textturnr @]}) is an
open-source software project that seeks to help course instructors
create learning materials.  The materials take the form of interactive
web pages and high quality PDF documents.  While these formats are
very different, with Ximera an author is able to separate content from
deployment, and write the source for both types of materials
simultaneously.

\subsection{Philosophy}

Since Ximera is built on \LaTeX\ source, we want to use \LaTeX\ as a
method of validating the code authors write. Hence, if you want to
write a Ximera online activity, the first step is \textbf{compiling}
\LaTeX\ documents.

Once you have \textbf{compiled} the \LaTeX\ documents, and you have
checked them for typos, accuracy, etc, the fact that they compile
should be reasonable evidence that they will display correctly in
Ximera.





\section{Basic workflow}

An author writes content as a \LaTeX\ document.  This produces a PDF
handout that can be distributed to students.  Next the author uploads
the same \LaTeX\ file to \link[github.com]{http://github.com}, a free
web-based service providing a number of features to developers and
authors.  In turn \link[github.com]{http://github.com} delivers the
file to the \link[Ximera]{http://ximera.osu.edu} interpreter, which
posts the file on the web.  The web page has essentially the same
content as the handout.  However, the web page typically has
interactive features not possible in the handout due to the physical
limitations of paper.  For example, the web page might pose a question
that if answered incorrectly, would offer hints or further questions
to the student.  This process is illustrated in the figure below.

\begin{image}
\includegraphics[scale=.25]{./XimeraGraphic.png}
\end{image}

\section{Trying it out}

At this point you need to make a decision, do you want to try out
Ximera on CoCalc (a cloud-based solution) or do you want to install
on your own machine?  If you just want to try this out with as little
investment as possible, CoCalc may be the way to go. The downside is
that if you become serious about developing in Ximera, you will
probably want to use your own machine.

\section{Examples of Ximera courses}

Perhaps the best way to learn is by looking at examples. We recommend
looking at \link[mooculus]{https://ximera.osu.edu/mooculus} and the
associated GitHub repository \link[mooculus/calculus]{https://github.com/mooculus/calculus}.

\end{document}
