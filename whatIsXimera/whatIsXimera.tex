\documentclass{ximera}
\title{What is Ximera?}
\outcome{Understand what Ximera is and how it can be used}
\begin{document}
\begin{abstract}
An introduction to the Ximera system.
\end{abstract}
\maketitle

\link[Ximera]{http://ximera.osu.edu} is an open-source software
project that seeks to help course instructors create learning
materials for their students in the form of interactive web pages and
high quality PDF documents.  Our strategy to achieve this goal is to
separate content from deployment.

An author writes content as a \LaTeX\ document.  This produces a PDF
handout that can be distributed to students.  Next the author uploads
the same \LaTeX\ file to \link[github.com]{http://github.com}, a free
web-based service providing a number of features to developers and
authors.  In turn \link[github.com]{http://github.com} delivers the
file to the \link[Ximera]{http://ximera.osu.edu} interpreter, which
  posts the file on the web.  The web page has essentially the same
  content as the handout.  However, the web page typically has
  interactive features not possible in the handout due to the physical
  limitations of paper.  For example, the web page might pose a
  question that if answered incorrectly, would offer hints or further
  questions to the student.  This process is illustrated in the figure
  below.

\begin{image}
\includegraphics[scale=.25]{./XimeraGraphic.png}
\end{image}

\subsection{Benefits of Ximera}
One benefit of \link[Ximera]{http://ximera.osu.edu} is that it
provides an easy way to produce interactive online materials.  Another
benefit is that many educators, particularly mathematicians, are
already quite familiar with \LaTeX\ and even find it easy to use.  And
because the \TeX\ language is extremely static in comparison with
other programming and markup languages, authors can expect the
\link[Ximera]{http://ximera.osu.edu} materials they create to be
usable in some form for the foreseeable future.

\subsection{Differences between Ximera and Learning Management Systems}
A Learning Management System (LMS) such as Blackboard or Moodle
provides students with a central webpage from which they can navigate
to the webpage of each of their courses, these course web pages all
formatted and laid out in exactly the same way.  Because of the
possibility of conveying grades to students, an LMS requires students
and instructors to have accounts, which are typically set up by the
college or University, as are the course web pages themselves.

By far, the most common way to use an LMS in our experience is to make
announcements or to distribute files to students.  While this could be
accomplished through email or simple web pages, the LMS, being largely
set up by the college or University, provides an even easier way to
distribute files and communicate with students.  It also organizes
students' course materials in a single location.

The function of an LMS most relevant to our discussion of
\link[Ximera]{http://ximera.osu.edu} is assessment.  Instructors
compose quizzes or homework assignments, which consist of sequences of
questions of various types.  Typically each question has a unique
answer, allowing the LMS to score the assignment and record grades in
a gradebook, another function of an LMS.  It should be born in mind
that composing questions for an LMS, particularly multiple-choice
questions, creates a lot of work for instructors. For this reason
textbook publishers are increasingly bundling their books with
accounts on proprietary LMS's, from which an instructor can select
precomposed exercises for students to complete.

In contrast \link[Ximera]{http://ximera.osu.edu} provides a more
flexible environment for creating course materials.  While an activity
on an LSM consists solely of a sequence of problems, a
\link[Ximera]{http://ximera.osu.edu} activity can contain whatever
content a \LaTeX\ document can contain, and in addition, can include
interactive questions of various types.

\subsection{Ways Ximera can be used}
Whereas an LMS typically only supplements a traditional course, an
entire course can be delivered through
\link[Ximera]{http://ximera.osu.edu}.  For example, a course could
consist of a sequence of \link[Ximera]{http://ximera.osu.edu}
activities, each of which primarily composed of text for students to
read, with an occasional question sprinkled in to confirm and deepen
students' understanding of the material presented just before the
question.  More substantial questions could appear at the end of the
activity.

Another possibility is to use \link[Ximera]{http://ximera.osu.edu} in
conjunction with a traditional lecture-based course, using
\link[Ximera]{http://ximera.osu.edu} to deliver pre-lecture
reading materials and exercises to be completed by students before
arriving to their lecture.  These could serve as a prelude to the
lecture, perhaps presenting background topics or materials to review
before the lecture.

The emphasis of \link[Ximera]{http://ximera.osu.edu}
is more on content delivery than on assessment.
We understand however that many students function
better knowing that their actions will be acknowledged by
their instructors. We therefore envisage providing ways
for instructors to access student scores on 
\link[Ximera]{http://ximera.osu.edu} activities
in the near future. In the meantime it might
be better to use a traditional LMS to deliver exercises
if keeping track of student participation is important.

\end{document}
