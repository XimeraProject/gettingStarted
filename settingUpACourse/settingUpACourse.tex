\documentclass{ximera}
%\def\sectionautorefname~#1\null{\S#1\null}
%\def\subsectionautorefname~#1\null{\S#1\null}
\title{Setting up a course}
%\def\itemautorefname~#1\null{#1\null}
\begin{document}
\begin{abstract}
Instructions for setting up a course.
\end{abstract}
\maketitle


\subsection{Setting up a Ximera course}



Now that you can create a course. A
\link[Ximera]{http://ximera.osu.edu} course consists of a directory
(\verb!anExampleCourse!)  containing a set of directories
(\verb!theFirstActivity!, \verb!theSecondActivity!) that all contain
\LaTeX\ files using the \verb!ximera! document class. This directory
also contains one or more \LaTeX\ files using the \verb!xourse!
document class, each of which is a launch page for the activities in
the course(s). This section describes how to create a course.
\begin{enumerate}

     
\item\label{Mkdir} 
In this example, we will create a directory called
\verb!anExampleCourse! that contains a directory called \verb!theFirstActivity!. We will also create the \verb!anExampleCourse.tex! file.
Open a terminal session and type the following commands to do that.
\begin{verbatim}
mkdir anExampleCourse
cd anExampleCourse 
mkdir theFirstActivity
touch anExampleCourse.tex
\end{verbatim}

\item Next we will modify the \verb!anExampleCourse.tex! file in the (\verb!anExampleCourse!) directory. Click ``Files," click on the (\verb!xandbox!) directory and click on the  (\verb!anExampleCourse!) directory, then click the file \verb!anExampleCourse.tex!. Copy and paste the following in the left-hand side of the window and click ``Save." It may complain that you haven't created the \verb!theFirstActivity.tex! file yet, but that's fine.

    

\begin{verbatim}
\documentclass{xourse}

\title{An example course}
%% This is the Name of your course 
%% personalize it

\begin{document}
\begin{abstract} %% This describes your course
This is a Ximera activity explaining how to get 
started with Ximera for course instructors.
\end{abstract}
\maketitle

%% Here we have a listing of the activities. 
\activity{theFirstActivity/theFirstActivity}

\end{document}
\end{verbatim}

Now return to the terminal window, which should still be in the directory \verb!theExampleCourse!. You need to add the file (and the directory containing it) to git and commit your changes, and also tell it who you are. \begin{verbatim}

git add anExampleCourse.tex
git commit -m "this is my first course"
\end{verbatim}

Every time you add or change files, you will need to run \verb!git add! and \verb!git commit -m "short description of changes"! to commit the changes to the server. The description of changes is necessary to commit, as is the \verb! -m!. If you are changing multiple files, you can use \verb!git add *!.


\begin{remark}
In general the file with the \verb!xourse! document class specifies
course information such as the name of the course, a description of
the course, and the names of all \LaTeX\ activity files comprising the
course, in the order they should be presented to students.  In
addition to a name and a description, \verb!anExampleCourse.tex!
  above specifies that there is one activity file
  \verb!theFirstActivity.tex!, written with or without the extension
  \verb!.tex!, and located in a directory called
  \verb!theFirstActivity!.  We will create this file and directory in
  the following step.

Generally courses should contain more than one activity. We recommend
placing each activity in a directory of the same name.  This
facilitates sharing activities among collaborators and makes reusing
existing activities easier.
%Later in this course, we will see examples of
%how to borrow existing activities from other courses
%rather than starting from scratch. 
We also recommend that the directory and the \LaTeX\ file have exactly
the same name as the title of the activity,
with all spaces removed and all words other than the first word
capitalized. So for example, if the title of the
activity were \verb!Plants native to Ohio!
the \LaTeX\ file \verb!plantsNativeToOhio.tex!
would be located in a directory called
\verb!plantsNativeToOhio!.
\end{remark}

\item Now create your first activity. In your terminal, move to the \verb!theFirstActivity!
directory and create a file called \verb!theFirstActivity.tex!.
This can be accomplished by
executing the commands below.
\begin{verbatim}

cd theFirstActivity 
touch theFirstActivity.tex
\end{verbatim}

\item\label{FirstExercise}
Click ``Files" again and navigate to the (\verb!theFirstActivity!) folder, open \verb!theFirstActivity.tex!,
and paste in the following text. Then save the file.
\begin{verbatim}
\documentclass{ximera}
\title{The First Activity}
\begin{document}
\begin{abstract}
This activity deals with Ximera activities.
\end{abstract}
\maketitle
\end{document}
\end{verbatim}

\begin{remark}
An activity should be composed as a regular
\LaTeX\ file in the document class \verb!ximera!.
It should contain the title of the activity and an abstract.
These will both appear on the course website in the navigation area,
so the abstract should be short.
At this stage your activity contains
a title and an abstract, but is otherwise blank.
\end{remark}
\item Now add this file to git and commit your changes. Back in the terminal, change to the directory \verb!anExampleCourse!
 and execute the following commands
\begin{verbatim}
git add theFirstActivity.tex
git commit -m "Added first activity file"
\end{verbatim}




\item Type \verb!xake bake! to compile the tex documents, then \verb!xake frost! to create a publication commit on top of your source commit. Finally type \verb!xake serve! to share your content with the world. For instance, my content will appear at https://ximera.osu.edu/turnloon/anExampleCourse


\item This is a good point to add some further content in the
  form of a simple exercise.  Update the file
  \verb!theFirstActivity.tex!  you created above so that it looks like
  the following.

\begin{verbatim}
\documentclass{ximera}
\title{The First Activity}
\begin{document}
\begin{abstract}
This activity deals with Ximera activities.
\end{abstract}
\maketitle
This activity is about creative work.
\begin{exercise}
  Choose the best place to work on mathematics.
  \begin{multipleChoice}
    \choice{At the library}
    \choice[correct]{At the caf\'e}
    \choice{In your office}
  \end{multipleChoice}
\end{exercise}
\end{document}
\end{verbatim}
\begin{remark}
The edits above insert
a multiple-choice question into the \verb!theFirstActivity!
activity. See the {\sf Question and answer types}
activity later in this tutorial
for more information on creating exercises.
\end{remark}

\item Change to the directory \verb!anExampleCourse!
  and execute the following commands.
\begin{verbatim}
git add theFirstActivity.tex
git commit -m "Added an exercise"
git push
xake bake
xake frost
xake serve
\end{verbatim}
If everything went well, you should see a URL printed on the terminal.
If not, see the {\sf Troubleshooting} activity in this tutorial or
send your questions to
\link[ximera@math.osu.edu]{mailto:ximera@math.osu.edu}.
\begin{remark} The commands above inform 
\link[git]{http://git-scm.com} that changes have been made to your
repository and communicates them to
\link[github.com]{http://github.com} which in turn communicates them
to \link[ximera.osu.ed]{http://ximera.osu.edu}.  You should execute
similar commands whenever you change files in your repository.
\end{remark}
\end{enumerate}

\subsection{Creating further activities}
From here you can create further \link[Ximera]{http://ximera.osu.edu}
activities as in step~\autoref{FirstExercise}.  You should issue a
\verb!git add! command after creating a new file or directory and a
\verb!git commit!  command followed by a \verb!git push! command
periodically to transmit your most recent changes to
\link[github.com]{http://github.com}.  You should also add the name of
your activity file to the \verb!xourse! file in the position relative
to other activities where you want the activity to appear.  Observe
however that once the filename appears in the \verb!xourse!  file the
corresponding activity will appear to students. It might therefore be
preferable to create a separate branch on GitHub until the activity is
ready for students.  During the editing phase you still view the
activity by processing it with \LaTeX\ and inspecting the resulting
PDF file, which might be helpful in any case for finding and
correcting mistakes.

\subsection{Other ways to set up a Ximera repository}\label{ForkClone}
There are other ways to create a \link[Ximera]{http://ximera.osu.edu}
course.  One possibility is to begin by creating the repository on
\link[github.com]{http://github.com}.  Then instead of executing the commands to
initialize the local copy of the repository, you could {\em clone} the
copy on \link[github.com]{http://github.com} using a \verb!git clone!
  command.  Alternately you could {\em fork} an existing repository,
  either your own or someone else's.  See the \link[git
    manual]{http://git-scm.com} for more information about the
  \verb!clone! and \verb!fork! commands.  Both possibilities above
  obviate step \autoref{Mkdir} since cloning or forking a
  \link[git]{http://git-scm.com} repository creates a local directory
  and initializes it as a \link[git]{http://git-scm.com} repository.



\end{document}
