\documentclass{ximera}

\title{Basic Git commands}

\begin{document}
\begin{abstract}
  Some basic Git commands.
\end{abstract}
\maketitle

\section{Creating a Git repository}

If you already have a git repository made, you can skip this
step. Otherwise you need to set-up a git repository.



\subsection{Creating the repository in the command line}

Go to GitHub, click ``+'' and Create New Repository Give it a title,
and a short description.

Ignore the ``initialize this repo with a README''

Click ``Create repository''

Now we will push your current directory to this repo. Once you are within your repo:

\begin{verbatim}
git init
git add .
git commit -m 'first push'
git remote add origin git@github.com:YOUR-GIT-USER/REPO-NAME.git
git push -u origin master
\end{verbatim}

where YOUR-GIT-USER is your GitHub username and REPO-NAME is the name
of the repository you created.


\subsection{Creating the repository via a client}

Drag and drop you new folder on to the (probably empty) repository
window. Then push to GitHub with the button in the upper right hand
corner. You want to always allow GitHub to use your SSH KeyChain.

You can confirm a repository was made by going to GitHub and viewing
your user profile. To add new files to the repository, go to the
``Repository'' heading in the Git client and click ``open in finder.''
From there one can add files by dragging and dropping, and then
syncing.


\section{Pushing Files}

Unless you like \verb|vi| you will want to set your basic editor. I
suggest doing:

\begin{verbatim}
git config --global core.editor "nano"
\end{verbatim}

When working with git, you will want to
  \begin{enumerate}
\item You'll need to \verb!pull! your new repo. Create a file. Suppose its name is:
  \verb|first.tex|
\item Type \verb!git add first.tex! to stage the changes you've made
  to the first.tex file. Or do \verb|git add -u| to add all updated
  files.
\item Type \verb!git commit -m "My first edit"! to commit the staged changes. You can also just do: \verb|git commit|
  \end{enumerate}




\end{document}
