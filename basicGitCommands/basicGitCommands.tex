\documentclass{ximera}

\title{Basic Git commands}

\begin{document}
\begin{abstract}
  Some basic Git commands.
\end{abstract}
\maketitle

If you already have a git repository made, you can skip this
step. Otherwise you need to set-up a git repository. This is not
difficult, but it is dependent on what you name your git
repository. See
\link[this]{https://help.github.com/articles/creating-a-new-repository/}

Unless you like \verb|vi| you will want to set your basic editor. I suggest doing:

\begin{verbatim}
git config --global core.editor "nano"
\end{verbatim}

When working with git, you will want to
  \begin{enumerate}
\item You'll need to \verb!pull! your new repo. Move ``first.tex'' into your repo. 
\item Type \verb!git add first.tex! to stage the changes you've made
  to the first.tex file. Or do \verb|git add -u| to add all updated
  files.
\item Type \verb!git commit -m "My first edit"! to commit the staged changes. You can also just do: \verb|git commit|
  \end{enumerate}




\end{document}
