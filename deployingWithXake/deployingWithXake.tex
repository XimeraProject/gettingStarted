\documentclass{ximera}

\title{Deploying with Xake}

\begin{document}
\begin{abstract}
Using Xake to deploy.
\end{abstract}
\maketitle

The program \verb|xake| is the Ximera ``make'' program.

\section{One-time setup}

Here is a list of commands that need to be run for the setup of a course. 

\begin{enumerate}
\item If you have a GPG key, do:
\begin{verbatim}
gpg --list-keys
\end{verbatim}
\item If you don’t already have a GPG key, then do
\begin{verbatim}
gpg --gen-key
\end{verbatim}
answer the questions, but \textbf{leave the passphrase blank} and copy
the long hex string as YOUR-GPG-KEY-ID ( ABCD3562DBF9929292 or
whatever) If you are using MacOS, you might not be able to leave the
passphrase blank. In this case go to \verb|https://gpgtools.org/| and
install the GPG Suite.  This will provide a GUI that will produce a
GPG key with spaces. Delete these spaces and this new key (without
spaces) is your key.

\begin{verbatim}
gpg --keyserver hkps://ximera.osu.edu/ --send-key YOUR-GPG-KEY-ID
\end{verbatim}
Now do
\item Now do: 
\begin{verbatim}
xake -k YOUR-GPG-KEY-ID name yourxakename
\end{verbatim}
You will need to enter your passkey; prompt may be broken but try typing it and hitting enter anyway

\item You may get vague error messages here if you have not successfully set up your git name and email; you can do this with:
\begin{verbatim}
git config --global user.name "YOUR-NAME"
git config --global user.email "YOUR-EMAIL"
\end{verbatim}
\end{enumerate}


\section{Work flow}

The commands above are simply for setup. On a regular basis, the work flow is much simpler.

You must work in the top level of your git directory.
\begin{verbatim}
git pull
git add -u
git commit
git push
xake bake
xake frost
xake serve
\end{verbatim}




  

\end{document}
