\documentclass{ximera}
\title{Installing locally: MacOS}

\begin{document}
\begin{abstract}
Instructions for installing Ximera locally.
\end{abstract}
\maketitle


\section{Install ximeraLaTeX}


Visit the repo on Github.com \link[XimeraProject/ximeraLatex GitHub
  page]{https://github.com/XimeraProject/ximeraLatex} and click the
``Clone in Desktop'' button.


Create the directory structure:

\begin{verbatim}
~/Library/texmf/tex/latex/
\end{verbatim}

and move ximeraLatex to \verb|~/Library/texmf/tex/latex/|. This will allow
all of your documents to find ximera.cls. Once you move ximeraLatex,
you will need to tell your GitHub client where to look for it, you can
do this using ``find files.''





\section{Install xake}

Xake is Ximera's version of ``make'' and it converts the
\LaTeX\ source into HTML.

\begin{enumerate}
\item First you need \link[HomeBrew]{https://brew.sh}. The following command, typed in a terminal window
\begin{verbatim}
/usr/bin/ruby -e "$(curl -fsSL https://raw.githubusercontent.com/Homebrew/install/master/install)"
\end{verbatim}
Should be all you need. 
\item Then you need to install \link[Go]{http://www.golangbootcamp.com/book/get_setup}. The following command, typed in a new terminal window
\begin{verbatim}
brew install go --cross-compile-common
\end{verbatim}
again should be all you need.
\item Install \verb|pkg-config| and \verb|libgit2|. This can be done via HomeBrew:
\begin{verbatim}
brew install pkg-config
brew install libgit2
brew install gpg
brew install mupdf
\end{verbatim}
If you get an error \verb|Error: An unsatisfied requirement failed this build| just follow directions given. 


\item Now perform the following commands in terminal:
\begin{verbatim}
mkdir -p ~/go/src/github.com/ximeraproject
export GOPATH=$HOME/go
cd ~/go/src/github.com/ximeraproject
git clone https://github.com/XimeraProject/xake.git
brew install ximeraproject/xake/xake
\end{verbatim}
\end{enumerate}
\end{document}
