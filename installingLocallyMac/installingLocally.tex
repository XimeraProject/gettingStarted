\documentclass{ximera}
\title{Installing locally: MacOS}

\begin{document}
\begin{abstract}
Instructions for installing Ximera locally.
\end{abstract}
\maketitle


\section{Install ximeraLaTeX}

This can be done by going to \link[XimeraProject/ximeraLatex GitHub
  page]{https://github.com/XimeraProject/ximeraLatex}, and scrolling
down to the directions for your particular platform.




\section{Install xake}

Xake is Ximera's version of ``make'' and it converts the
\LaTeX\ source into HTML.

\begin{enumerate}
\item First you need \link[HomeBrew]{https://brew.sh}. The following command, typed in a terminal window
\begin{verbatim}
/usr/bin/ruby -e "$(curl -fsSL https://raw.githubusercontent.com/Homebrew/install/master/install)"
\end{verbatim}
Should be all you need. 
\item Then you need to install \link[Go]{http://www.golangbootcamp.com/book/get_setup}. The following command, typed in a new terminal window
\begin{verbatim}
brew install go --cross-compile-common
\end{verbatim}
again should be all you need.
\item Install ''pkg-config” and ''libgit2”. This can be done via HomeBrew:
\begin{verbatim}
brew install pkg-config
brew install libgit2
brew install gpg
brew install mupdf
\end{verbatim}
\item Now perform the following commands in terminal:
\begin{verbatim}
mkdir -p ~/go/src/github.com/ximeraproject
export GOPATH=$HOME/go
cd ~/go/src/github.com/ximeraproject
git clone https://github.com/XimeraProject/xake.git
brew install xake/xake
\end{verbatim}
\end{enumerate}
\end{document}
