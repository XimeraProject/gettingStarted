\documentclass{ximera}
\title{Installing Locally: Linux}

\begin{document}
\begin{abstract}
Instructions for installing Ximera locally.
\end{abstract}
\maketitle



\section{Install ximeraLaTeX}

This can be done by going to \link[XimeraProject/ximeraLatex GitHub
  page]{https://github.com/XimeraProject/ximeraLatex}, and scrolling
down to the directions for your particular platform.




\section{Install xake}

Xake is Ximera's version of ``make'' and it converts the
\LaTeX\ source into HTML.

\subsection{Installing xake on Arch}

\verb!yaourt -S xake-git! should work assuming you use \texttt{yaourt}.

  
\subsection{Installing xake on Ubuntu}
\begin{verbatim}
sudo apt-get install golang-go
\end{verbatim}
Corresponding libgit2 and git2go installations are needed:
\begin{verbatim}
sudo apt-get install libgit2-dev
sudo apt-get install golang-git2go-dev
mkdir -p ~/go/src/github.com/ximeraproject
export GOPATH=$HOME/go
cd ~/go/src/github.com/ximeraproject
git clone https://github.com/XimeraProject/xake.git
cd ~/go/src/github.com/libgit2/git2go
git checkout v24
cd ~/go/src/github.com/ximeraproject/xake
go get .
go build .
Use command ./xake or better yet, add xake to PATH by...
Also set your PATH so that ~/go/bin/ is in it: (edit .profile)
PATH=$PATH:~/go/bin
\end{verbatim}
Reboot!  Now you should be able to use the command xake

\end{document}
