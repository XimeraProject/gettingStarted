\documentclass{ximera}
\title{Installing locally: Linux}

\begin{document}
\begin{abstract}
Instructions for installing Ximera locally.
\end{abstract}
\maketitle



\section{Install ximeraLaTeX}

This can be done by going to \link[XimeraProject/ximeraLatex GitHub
  page]{https://github.com/XimeraProject/ximeraLatex}, and scrolling
down to the directions for your particular platform.




\section{Install xake}

Xake is Ximera's version of ``make'' and it converts the
\LaTeX\ source into HTML.

\subsection{Installing xake on Arch}

\verb!yaourt -S xake-git! should work assuming you use \texttt{yaourt}.

  
\subsection{Installing xake on Ubuntu}

Download the \verb|.deb| file from here:

\link{https://github.com/XimeraProject/xake/releases}

Then in the directory, do

\begin{verbatim}
sudo apt install ./xake_0.9.2_amd64.deb.deb
\end{verbatim}

where the \verb|0.9.2| will be replaced with the version of the
\verb|.deb| file you have downloaded.

\subsection{Installing xake on Red Hat}

Download the \verb|.rpm| file from here:

\link{https://github.com/XimeraProject/xake/releases}

And install as usual. 
\end{document}
